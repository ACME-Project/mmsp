% MMSP.chapter2.tex

\chapter{Getting started with {\tt MMSP}}
The following sections provide the necessary information for new users to obtain and set up {\tt MMSP}.  Typically, this involves downloading a source code archive, unpacking it, and running a few tests.  Developers or those interested in maintaining an up-to-date version of the code should consider checking out a copy from the {\tt subversion} repository.

\section{Downloading {\tt MMSP}}
The {\tt MMSP} source code is hosted online at MatForge (http://matforge.org).  From the main MatForge site, users should navigate to the {\tt MMSP} home page and choose the appropriate links under the section ``Download,''  as these links are set up to point to the latest {\tt MMSP} release.  Archives containing the {\tt MMSP} source code are provided in the usual Linux ``tarball'' convention (with filename extension~{\tt .tar.gz}), and as a ``zip'' file (with extension~{\tt .zip}).  Users should download an archive that can be unpacked by existing utilities on their platform; if you are unsure which archive to download, read the following section titled ``Installing {\tt MMSP}'' before making your choice.

Alternatively, a user may choose to check out a copy from the {\tt subversion} repository.  This, of course, requires the {\tt subsversion} version control software (http://www.subversion.com) to be installed on the target machine, as well as a working internet connection.  The command to issue is
\begin{shadebox}
\begin{verbatim}
    svn+ssh://www.matforge.org/usr/local/svn-cmu-repos/MMSP
\end{verbatim}
\end{shadebox}
If this is successful, then there is no need to perform the actions described below in Installation, and the user should move on to Setup.  Having a local copy of {\tt MMSP} makes it simple to keep your code up-to-date.  From the root folder, simply type
\begin{shadebox}
\begin{verbatim}
    svn update
\end{verbatim}
\end{shadebox}
to update a working copy with the latest version of the {\tt MMSP} source code.  See the {\tt subversion} documentation for more details.

\section{Installing {\tt MMSP}}
After an appropriate source code archive is obtained as described above, the next step is to install {\tt MMSP}.  This should be as simple as unpacking the archive.  Users with administrator priveledges may choose to install the {\tt MMSP} header files in a location that will be searched by their compiler's preprocessor, but we do not describe how to do this here.  The following paragraphs provide platform-specific instructions for a local installation.

\paragraph{Linux/Unix}
Local installation for Linux users should simply involve unpacking the archive.  As most Linux systems have means to unpack both tarballs and zip files, there is likely no reason to prefer either.  After downloading the archive file, move it to the directory where you want {\tt MMSP} to reside, making sure that you have read access to the file as well as write access to the directory.  Then issue a command to unpack the archive, e.g.
\begin{shadebox}
\begin{verbatim}
    tar zxf MMSP.3.0.6.tar.gz
\end{verbatim}
\end{shadebox}
or
\begin{shadebox}
\begin{verbatim}
    unzip MMSP.3.0.6.zip
\end{verbatim}
\end{shadebox}
This will unpack the contents of the archive into a folder named {\tt MMSP}.  Next, type
\begin{shadebox}
\begin{verbatim}
    ls MMSP
\end{verbatim}
\end{shadebox}
which should indicate that folders such as {\tt MMSP/doc}, {\tt MMSP/examples}, etc.~have been created.  If either command fails or the folder {\tt MMSP} is not created, check the {\tt tar} or {\tt unzip} documentation.

\paragraph{Mac OS}
Mac OS users will follow much of the same procedure as Linux users, so it is advisible to read the previous section on Linux installation.  For those uninitiated, or who have never had any previous reason to use it, the {\tt Terminal} application can be found under {\tt Applications/Utilities}.  Again, all that said above for Linux installation should apply.

\paragraph{MS Windows}
For those who insist on using MS Windows, it is still possible to use {\tt MMSP}.  The preferred way to do this is to use the {\tt cygwin} environment (http://www.cygwin.com).  To use {\tt cygwin}, it is necessary that appropriate packages, such as {\tt gcc} (the GNU compiler) and {\tt make} (the GNU make utility), have been installed.  These are optional packages that must be chosen manually during installation.  If {\tt cygwin} has been installed properly, {\tt MMSP} may be installed as described above for Linux installations.  It is also possible to compile {\tt MMSP} source code within a code development environment such as Visual Studio, however, {\tt MMSP} code is typically so simple that any code management beyond command line or makefile compilation is only a hinderance.

\section{Setting up {\tt MMSP}}
Once {\tt MMSP} has been installed, there are a few useful tests and utility programs that should be generated.  First, enter the {\tt MMSP/test} directory and type
\begin{shadebox}
\begin{verbatim}
    make test
\end{verbatim}
\end{shadebox}
If this compiles with no problems, then you're in luck; issue the command
\begin{shadebox}
\begin{verbatim}
    ./test
\end{verbatim}
\end{shadebox}
which will generate a message indicating successful operation.  If the test program fails to compile, it is most likely because either {\tt make} or {\tt g++} (the GNU {\tt c++} compiler) is not installed on the system or is not configured properly.  Of course, any other ISO-compliant {\tt c++} compiler may be used instead.  If there is a problem at this stage, users should check their configuration.

Those who plan to use {\tt MMSP} with the MPI (Message Passing Interface) libraries should also type
\begin{shadebox}
\begin{verbatim}
    make parallel
\end{verbatim}
\end{shadebox}
which produces a parallel version of the test program.  If this is successful, then run the program using an appropriate command for your MPI distribution, e.g.\
\begin{shadebox}
\begin{verbatim}
    mpirun -np 4 parallel
\end{verbatim}
\end{shadebox}
Note that for successful compilation, the MPI distribution is expected to provide a compilation script named {\tt mpic++} and a the header file named {\tt mpi.h}.  If this program fails to compile, it may be that the user's MPI distribution provides {\tt mpicxx}, {\tt mpiCC}, or the like instead of {\tt mpic++}, or that it provides {\tt mpicxx.h} or the like instead of {\tt mpi.h}.  In this case, the user should edit the {\tt Makefile} accordingly.  Likewise, the appropriate command to run the compiled program may differ depending on the MPI distribution.  This may take the form of, e.g.\ {\tt mpiexec}, instead of {\tt mpirun}.  Unfortunately, MPI distributions do not adhere to a single standard with respect to compiling and running parallel programs, and so it is largely left to the user to determine what must be done for their particular system.

Finally, enter the directory {\tt MMSP/utility} and type
\begin{shadebox}
\begin{verbatim}
    make utility
\end{verbatim}
\end{shadebox}
This will produce a number of conversion programs.  In particular, it will produce several programs such as {\tt mmsp2vti} which can be used to convert {\tt MMSP} grid data files into formats that can be read by visualization software such as ParaView.  Because the programs provided in this directory are used so often, {\tt MMSP} users may wish to add the {\tt MMSP/utility} directory to their command path.  This can be achieved by adding the following lines to their {\tt \$HOME/.bashrc} file,
\begin{shadebox}
\begin{verbatim}
    PATH=$PATH:MMSP/utility
    export PATH
\end{verbatim}
\end{shadebox}
for users of the {\tt bash} shell, or
\begin{shadebox}
\begin{verbatim}
    setenv PATH $PATH:MMSP/utility
\end{verbatim}
\end{shadebox}
for users of the {\tt tcsh} shell.


