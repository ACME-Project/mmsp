% MMSP.chapter3.tex

\chapter{\MMSP tutorials}
The following sections present a few short tutorials on writing, compiling, and running simple \MMSP programs.  For those totally confused by the syntax of the examples presented here we suggest consulting an introductory {\tt c++} text first.

\section{A quick tutorial}
\subsection{The {\tt Hello MMSP!} program}
Because every good programming language or interface tutorial starts with a ``Hello World!'' example program, we'll do the same.  For most \MMSP applications, we include a header file named {\tt MMSP.hpp}.  Then we need a {\tt main()} program and a few lines to print out our message.  Here it is:
\begin{shadebox}
\begin{verbatim}
    #include"MMSP.hpp"

    int main(int argc, char* argv[])
    {
        MMSP::Init(argc,argv);

        std::cout<<"Hello MMSP!"<<std::endl;

        MMSP::Finalize();
    }
\end{verbatim}
\end{shadebox}
The only code here that should look out of the ordinary are the statements
\begin{shadebox}
\begin{verbatim}
    MMSP::Init(argc,argv);
\end{verbatim}
\end{shadebox}
and
\begin{shadebox}
\begin{verbatim}
    MMSP::Finalize();
\end{verbatim}
\end{shadebox}
What do these lines do?  For single processor programs, they do absolutely nothing -- they could actually be removed without any consequences.  However, for programs that use the message passing interface (MPI), they act as wrappers for the similarly named {\tt MPI::Init} and {\tt MPI::Finalize} commands.  It's useful to include them here because they'll allow us to write programs that may be compiled for both single or multiple processor environments.

Programmers familiar with {\tt c++} will notice that there's obviously some \MMSP namespace being used here.  For those less familiar, namespaces are a somewhat recent addition to {\tt c++} that are used as a means of avoiding naming conflicts.  We can avoid using namespace resolution so frequently if we use an appropriate {\tt using} statement, i.e. 
\begin{shadebox}
\begin{verbatim}
    #include"MMSP.hpp"
    using namespace MMSP;

    int main(int argc, char* argv[])
    {
        Init(argc,argv);

        // etc. 
    }
\end{verbatim}
\end{shadebox}
Namespaces serve to prevent programming errors and to ensure code reusability, so naturally we should apply {\tt using} statements with care.

The observant reader may also notice that we've used some of the stream input/output functions of {\tt c++} without including the requisite {\tt <iostream>} header.  In fact, this header and many other standard {\tt c++} headers are included implicitly thorough the file {\tt MMSP.hpp}.  If you need a particular standard header for your application and aren't sure if it has been included by {\tt MMSP.hpp}, you can always just {\tt \#include} it in your source code without any ill effects.

So that's it as far as the code goes.  This is a source file that may be compiled for both single and parallel simulations of... nothing.  In a moment we will move on to code that actually does something, but for now we should say a few words about compiling the code and running the executable.

\subsection{Compiling and running {\tt Hello MMSP!}}
Compiling code is a task that is, unfortunately, fairly platform dependent.  While \MMSP programs should compile easily on any platform, the required steps to do so may not look like the method shown here.  That said, let's look at how we would compile the previous example for a typical Linux or Unix setup.  Suppose the above code has been saved to a file named {\tt hello.cpp}, and that \MMSP has been extracted to the same directory.  A typical Linux machine will have at the least the GNU family of compilers installed, in which case we would type the command 
\begin{shadebox}
\begin{verbatim}
    g++ -I MMSP/include hello.cpp -o hello
\end{verbatim}
\end{shadebox}
which produces an executable file named {\tt hello}.  The compiler option {\tt -I ...} suggests a directory to search for non-standard headers; if your \MMSP headers reside somewhere else, you'll need to make the appropriate change.  To run the program, we type
\begin{shadebox}
\begin{verbatim}
    ./hello
\end{verbatim}
\end{shadebox}
which should produce our message.  Many other compilers on Linux and Unix machines use the same options as listed here, so this line may be very close to what you might use, even if you're not using {\tt g++}.

Now let's move on to parallel compilation.  We assume that if you're not skipping over this part, you have the MPI libraries installed on your machine (as well as your cluster).  With a typical MPI installation, a number of programs are included which effectively wrap your usual {\tt c++} compiler with a script named something like {\tt mpic++}, {\tt mpicxx}, {\tt mpicc}, etc.  With this in mind, we issue a command which may look like
\begin{shadebox}
\begin{verbatim}
    mpicxx -I MMSP/include -include mpi.h hello.cpp -o hello
\end{verbatim}
\end{shadebox}
which again produces an executable named {\tt hello}.  This time, however, we need to run {\tt hello} using an MPI command such as {\tt mpirun} or {\tt mpiexec} (see the documentation for your MPI distribution),
\begin{shadebox}
\begin{verbatim}
    mpirun -np 4 hello
\end{verbatim}
\end{shadebox}
which in this case produces our message four times.  The author sincerely hopes that your experience is this straightforward, but don't count on it!

\section{An example using the {\tt grid} class}
In this section, we look at an example that uses the \MMSP {\tt grid} template class, and actually resembles part of a program you might actually use:
\begin{shadebox}
\begin{verbatim}
    #include"MMSP.hpp"
    using namespace MMSP;

    int main(int argc, char* argv[])
    {
        Init(argc,argv);

        grid<2,int> GRID(1,0,100,0,100);

        for (int x=xmin(GRID); x<xmax(GRID); x++)
            for (int y=ymin(GRID); y<ymax(GRID); y++)
                GRID[x][y] = x+y;

        output(GRID,"testgrid");

        Finalize();
    }
\end{verbatim}
\end{shadebox}
This program performs the amazing feat of generating a new 2-dimensional {\tt grid} object, assigning to each node the sum of {\tt x} and {\tt y}, and then writing the final state of the {\tt grid} to a file, the name of which is simply {\tt testgrid}.  Of course this isn't really anything special, but the imaginative reader may begin to see how this program might be used as a template to generate other, more useful {\tt grid} objects.

Several new features have been introduced in this example.  First, the line
\begin{shadebox}
\begin{verbatim}
    grid<2,int> GRID(1,0,100,0,100);
\end{verbatim}
\end{shadebox}
constructs the {\tt grid} object.  Objects of the {\tt grid} class take two template arguments, the first being the dimension ({\tt 2}) and the second being the data type ({\tt int}).  In {\tt MMSP}, the dimension of a {\tt grid} can be any (constant) integer greater than zero, while the data type can be any of {\tt c++}'s built-in types or any of the \MMSP data types (to be discussed later).  In other words, \MMSP allows for simulations in arbitrary spatial dimensions, and the type of data stored at each {\tt grid} point is quite flexible.

The name of the {\tt grid} comes next ({\tt GRID}), followed by constructor arguments in parentheses.  These arguments indicate the total number of fields to create at each node (relevant only for vector-like data types) and the lower and upper limits in the {\tt x} and {\tt y} directions, respectively.  Note that these limits are the {\em global} limits, the portion of a {\tt grid} stored on a given machine may be smaller in parallel programs.  Also note that the limits may be either positive or negative, as long as the upper limit is always larger than the lower limit.

To give a sense of how the above generalizes to other dimensions and data types, consider the following:
\begin{shadebox}
\begin{verbatim}
    grid<3,double> GRID(1,0,128,0,256,0,256);
\end{verbatim}
\end{shadebox}
This statement declares a 3-dimensional {\tt grid} object with size $128\times256\times256$ and the built-in data type {\tt double}.

Returning to the example, the next lines iterate through the nodes of the {\tt grid},
\begin{shadebox}
\begin{verbatim}
    for (int x=xmin(GRID); x<xmax(GRID); x++)
        for (int y=ymin(GRID); y<ymax(GRID); y++)
            GRID[x][y] = x+y;
\end{verbatim}
\end{shadebox}
The functions {\tt xmin}, {\tt xmax} and {\tt ymin}, {\tt ymax} have been invoked to retrieve the {\em local} limits in each direction.  As suggested above, the local grid size in a parallel program is always smaller than the global grid size because parallelization in \MMSP is achieved through spatial subdivision.  Use of these functions ensures that we iterate only through the nodes stored by the local process.  While it isn't strictly necessary to use these functions in code meant to be run on a single processor, using them now will make parallelization completely trivial.

Also of note here is the fact that the subscript operator {\tt [ ]} works the same way for \MMSP grid objects as it does for subscripted arrays.  However, we will later see that the \MMSP subscript operators are ``smarter'' in the sense that they're cognizant of boundary conditions and adjust appropriately for calls made to nodes that are out of bounds.  There are also more advanced ways to access {\tt grid} data which we will discuss later.

Finally, the line 
\begin{shadebox}
\begin{verbatim}
    output(GRID,"testgrid");
\end{verbatim}
\end{shadebox}
seems fairly self-explanatory, and it is.  This writes the data contained within the {\tt grid} object to a file.  Here we also note that in the case of a parallel program, the {\tt output} function performs the additional task of pieceing back together the global {\tt grid} object from all local {\tt grid} objects before writing to a file.

As a point of reference, let's now look at how we might do something similar to the example above with the usual {\tt c} or {\tt c++} subscripted arrays:
\begin{shadebox}
\begin{verbatim}
    int main(int argc, char* argv[])
    {
        int GRID[100][100];

        for (int x=0; x<100; x++)
            for (int y=0; y<100; y++)
                GRID[x][y] = x+y;

        // etc.
    }
\end{verbatim}
\end{shadebox}
This may look considerably simpler, but consider the following: how do we run this code as a parallel process?  The immediate answer is that we introduce MPI funtion calls, but then we also need to decide how to subdivide the grid, how to deal with boundary conditions, how to put the pieces back together when we're done, etc.  And what if we want to change the number of processors or number of nodes in each direction and maintain an optimal subdivision scheme?  Starting from scratch, these operations all require a great deal of code development, whereas all of these are done by \MMSP automatically.


\section{More examples}
This section provides a number of additional examples demonstrating typical \MMSP operations.

\subsection{Accessing node data}
In the previous section, we saw how data within a {\tt grid} object may be accessed through the subscript operator, e.g.
\begin{shadebox}
\begin{verbatim}
    // assign some node of GRID2D to "value"
    double value = GRID2D[x][y];

    // assign "value" to some node of GRID3D 
    GRID3D[x][y][z] = value;
\end{verbatim}
\end{shadebox}
There are two other ways to access data, both of which may be useful depending on the context.  The first alternative is through integer subscripting using the parentheses operator {\tt ( )}, e.g.
\begin{shadebox}
\begin{verbatim}
    // assign the value of the ith node of GRID2D to "value"
    double value = GRID2D(i);

    // assign "value" as the value of the ith node of GRID3D
    GRID3D(i) = value;
\end{verbatim}
\end{shadebox}
How is this operator useful?  The answer is that it allows us to write code that works nicely for {\tt grid} objects of arbitrary dimension:
\begin{shadebox}
\begin{verbatim}
    for (int i=0; i<nodes(GRID); i++) GRID(i) = ... ;
\end{verbatim}
\end{shadebox}
Here, the function {\tt nodes} tells us how many nodes\footnote{The {\tt nodes} function does not count ``ghost'' cells when such exist.  This is not a problem because the parentheses operator skips these nodes anyway.} of the {\tt grid} object are stored locally, so that we can simply iterate through them with a single loop.  If we perform the same operations with each node regardless of dimension, then our code can be reused without adding additional loops, subscript operators, etc.

The second alternative is through {\tt vector}-based subscripting.  In this case, an {\tt MMSP::vector} object is used:
\begin{shadebox}
\begin{verbatim}
    MMSP::grid<2,double> GRID(1,0,100,0,100);
    ...

    for (int i=0; i<nodes(GRID); i++) {
        MMSP::vector<int> x = MMSP::position(GRID,i);
        GRID(x) = ... ;
    }
\end{verbatim}
\end{shadebox}
This example shows the declaration of an {\tt MMSP::vector} object, and its assignment to the coordinates of the $i^\text{th}$ node of the {\tt grid}.  How might this be useful?  Suppose we needed to perform some operation on all the nearest neighbors of a given node.  With {\tt vector}-based subscripting, we can use {\tt vector} addition to make quick work of the problem,
\begin{shadebox}
\begin{verbatim}
    for (int i=0; i<nodes(GRID); i++) {
        MMSP::vector<int> x = MMSP::position(GRID,i);

        for (int j=0; j<dim; j++) {
            MMSP::vector<int> dx(dim); dx[j] =  1;

            GRID(x+dx) = ... ;
            GRID(x-dx) = ... ;
        }
    }
\end{verbatim}
\end{shadebox}
While this kind of operation could be handled quite easily with the more traditional subscript operator, note that this code would be much easier to extend to any number of spatial dimensions.

\subsection{Boundary conditions}
Boundary conditions in \MMSP are handled automatically.  To show what this means, consider the following:
\begin{shadebox}
\begin{verbatim}
    grid<2,double> f(1,0,100,0,100);

    for (int x=xmin(f); x<xmax(f); x++)
        for (int y=ymin(f); y<ymax(f); y++) {
            double dfdx = (f[x+1][y]-f[x-1][y])/(2.0*dx(f));
            ...
        }
\end{verbatim}
\end{shadebox}
This code traverses each node in the {\tt grid}, computing the derivative along the $x$ axis, among a number of other possible operations.  This is simple enough, but what happens at the {\tt grid} boundaries?  If similar code was written using built-in arrays, it would typically fail with a segmentation fault, and in any case, it certainly wouldn't produce the desired results.  However, with {\tt MMSP} {\tt grid} objects, this isn't a problem; all of the data access operators in \MMSP are cognizant of domain boundaries and always enforce the chosen boundary conditions.

For the code example above, periodic boundary conditions are used.  This is the default boundary condition in \MMSP.  A number of other boundary conditions are possible, including Dirichlet, Neumann, and mirror boundaries.  Parallel processor boundaries are another special type of boundary condition which is not explicitly set by the user.

To change the boundary conditions used for a {\tt grid}, one might use\footnote{Note, as usual, the boundary conditions and ``set'' functions all belong to namespace \MMSP and must be used in the absence of a {\tt using} statement.}
\begin{shadebox}
\begin{verbatim}
    grid<2,double> GRID(1,0,100,0,100);

    // set x-axis boundary conditions
    b0(GRID,0) = Dirichlet;         // BC at lower x-axis boundary
    b1(GRID,0) = mirror;            // BC at upper x-axis boundary

    // set y-axis boundary conditions
    b0(GRID,1) = periodic;          // BC at lower y-axis boundary
    b1(GRID,1) = periodic;          // BC at upper y-axis boundary
\end{verbatim}
\end{shadebox}
This example demonstrates that the boundary conditions may not only be set independently for each coordinate axis, but that it's also possible to independently set the conditions at the upper and lower boundaries (within reason).



\section{The prototypical \MMSP program}
\MMSP was created with the intent that it would be used for mesoscale microstructure simulation.  So let's look at how we would typically write code to do just that.
\begin{shadebox}
\begin{verbatim}
    // prototype.cpp
    #include"update.hpp"
    using namespace MMSP;

    int main(int argc, char* argv[])
    {
        Init(argc,argv);

        grid<2,double> GRID(argv[1]);

        update(grid,atoi(argv[3]);

        output(GRID,argv[2]);

        Finalize();
    }
\end{verbatim}
\end{shadebox}
In this example, in addition to the usual \MMSP boilerplate code, we have a {\tt grid} constructor that reads from a file with a name specified by {\tt argv[1]}, a call to some function called {\tt update} (more on this in a moment), and a call to {\tt output} the {\tt grid} object to a file with a name specified by {\tt argv[2]}. After compiling this program, we would run it with the command line
\begin{shadebox}
\begin{verbatim}
    ./program initial.PF final.PF 100
\end{verbatim}
\end{shadebox}
or, if we compiled with the MPI libraries, we would run in with something similar to
\begin{shadebox}
\begin{verbatim}
    mpirun -np 4 program initial.PF final.PF 100
\end{verbatim}
\end{shadebox}
Here, $100$ is the intended number of time steps that the {\tt update} function will perform.

Now that we have our prototype {\tt main()} function, let's look at the {\tt update} function.  For this example, we would have a new header file named {\tt update.hpp},
\begin{shadebox}
\begin{verbatim}
    // update.hpp
    #include "MMSP.hpp"
    using namespace MMSP;

    void update(grid<2,double>& GRID, int steps)
    {
        grid<2,double> update(GRID);

        for (int step=0; step<steps; step++) {
            for (int x=xmin(GRID); x<xmax(GRID); x++)
                for (int y=ymin(GRID); y<ymax(GRID); y++) {

                    // replace with "real" computations...
                    update[x][y] = GRID[x][y];
                }

            swap(GRID,update);
            ghostswap(GRID);
        }
    }
\end{verbatim}
\end{shadebox}
Here we define the {\tt update} function, with its first argument a {\tt grid} object and its second argument the number of time steps to perform.  The only thing we should bring special attention to here is the use of the ampersand ({\tt \&}) to force call-by-reference, which overrides {\tt c++}'s usual convention of call-by-value (see a typical introductory text for more).

On the first line within the {\tt update} function, we create a new {\tt grid} object, itself called {\tt update}.  The new {\tt grid} is created using a constructor with {\tt GRID} as its argument.  What this does is it generates a {\tt grid} with the same spatial extent, number of fields, parallel topology, etc.  In other words, the {\tt grid} called {\tt update} is a suitable workspace to store the values of {\tt GRID} for the next time step as we compute them.

Next, we have a loop over all time steps.  Within this loop we iterate through the nodes of {\tt GRID} just as in the example of the previous section.  Again, note that the limiting values of {\tt x} and {\tt y} are obtained by appropriate function calls.  At each node, we would normally perform some meaningful computation to determine the value of {\tt GRID} at the next time step, then store the value in {\tt update}.  Here we simply copy the value of {\tt GRID} at this node to {\tt update}.

After computations are performed at each node, we have two more operations.  First, we {\tt swap} the data of {\tt GRID} and {\tt update}.  {\tt GRID} now contains the new values, while {\tt update}, the workspace, contains the old.  Next, a function called {\tt ghostswap} is called with {\tt GRID} as its argument.  For single processor programs, the {\tt ghostswap} function does nothing.  For parallel programs, it performs a coordinated series of ``ghost'' cell data swaps, such that the ghosts associated with the local portion of {\tt GRID} are filled with the appropriate data from all neighboring processors.  And again, while it's not critical to include this line when our intent is to write a single processor program, it will serve us well to include it here because it will allow us to easily parallelize the code later.



